\documentclass{notes}

\begin{document}
\title{SICP Notes}
\author{James Sungarda}
\date{\today}
\maketitle

\section{Programming Languages}
Programming languages combine simple ideas to complex ideas, they are composed of:
\begin{itemize}
    \item \textbf{Primitive expressions} - the simplest elements of a language, such as \emph{numbers and variables}.
    \item \textbf{Compound expressions} - combinations of primitive expressions that \emph{produce new values}, such as arithmetic operations.
    \item \textbf{Evaluation rules} - the rules that define how expressions are evaluated to produce values.
    \item \textbf{Environment} - the context in which expressions are evaluated, including variable \emph{bindings} and function \emph{definitions}.
\end{itemize}

\section{Evaluating methods}
\subsection{Substitution Model}
The substitution model is a method of evaluating expressions by replacing variables with their values. It involves:
\begin{itemize}
    \item Identifying the variables in an expression.
    \item Replacing each variable with its corresponding value.
    \item Simplifying the resulting expression until a final value is obtained.
\end{itemize}

\subsection{Normal Order Evaluation}
Normal order evaluation is a strategy for evaluating expressions where the outermost expressions are evaluated first. It involves:
\begin{itemize}
    \item Delaying the evaluation of expressions until their values are needed.
    \item Ensuring that expressions are evaluated in a way that avoids unnecessary computations.
    \item Allowing for the evaluation of expressions that may not terminate under other evaluation strategies. (See 2.2.1)
\end{itemize}
\textbf{Key idea} in this strategy is that it evaluates arguments \emph{only when they are needed}, which can lead to more efficient computations in some cases.

\subsubsection{Implications of each strategy}
Consider the following program
\begin{minted}{js}
function p() { return p(); }

function test(x, y) {
    return x === 0 ? 0 : y;
}
\end{minted}
In this example, the function \texttt{p} will lead to an infinite loop if evaluated, while the function \texttt{test} will return 0 if \texttt{x} is 0, otherwise it returns \texttt{y}. \\
The implications of evaluation strategies can lead to different outcomes based on how expressions are evaluated.

\begin{itemize}
    \item \textbf{Substitution Model} would lead to an infinite loop when evaluating \texttt{p()}.
    \item \textbf{Normal Order Evaluation} would also lead to an infinite loop for \texttt{p()} but would allow \texttt{test(0, 5)} to return 0 without evaluating \texttt{y}.
\end{itemize}

\section{Conditional Expressions}
Consider the following expression:
\begin{minted}{js}
p ? x : y
\end{minted}
\begin{itemize}
    \item \texttt{p} is a \textbf{predicate} that evaluates to either true or false.
    \item \texttt{x} is the \textbf{consequent} that is evaluated if \texttt{p} is true.
    \item \texttt{y} is the \textbf{alternative} that is evaluated if \texttt{p} is false.
\end{itemize}
The interpreter starts by evaluating the predicate \texttt{p}. \\ 
If \texttt{p} is true, it evaluates and returns \texttt{x}; if \texttt{p} is false, it evaluates and returns \texttt{y}. 

\subsection{Predicates}
Primitive predicates include \(>=\), \(>\), \(=\), \(<=\), and \(<\). These predicates are used to compare values and return a boolean result (true or false).
\subsubsection{Compound Predicates}
Compound predicates are formed by combining primitive predicates using logical operators such as \texttt{and}, \texttt{or}, and \texttt{not}. These operators allow for more complex conditions to be evaluated.

\section{Functions as building blocks}
Functions are \emph{black boxes} that take inputs and produce outputs, and suppresses the details of how they work. \\
The parameters of a function are local to the body of the function.

\subsection{Best practice}
Consider a function \texttt{sqrt(x)} which uses Newton's method to compute the square root of \(x\):\\
It contains several helper functions \texttt{\(average(x, y)\)} and \texttt{\(good\_enough(guess, x)\)}.
These helper functions should be defined \emph{within} the body of \texttt{sqrt(x)}.\\
The end-user need not need to know about these helper functions. 

\subsubsection{Block structures}
\begin{itemize}
    \item Any pair of braces designates a \emph{block}, and the nesting of declarations are called \emph{block structures}. 
    \item As the variable \texttt{x} is bound within the block of \texttt{sqrt(x)}, it is accessible by all functions inside it (and as such explicit passings are unnecessary), such practice is known as \textit{lexical scoping}.\\
\end{itemize}

\section{Functions}
\subsection{Linear Recursion and Iteration}
The factorial function can be defined in two ways:
\begin{minted}{js}
function factorial(n) {
    return n === 1 ? 1 : n * factorial(n - 1);
}
\end{minted}

or alternatively,
\begin{minted}{js}
function factorial(n) {
    return fact_iter(1, 1, n);
}
function fact_iter(product, counter, max_count) {
    return counter > max_count
           ? product
           : fact_iter(counter * product,
                       counter + 1,
                       max_count);
}
\end{minted}

Both of these structures are \emph{syntactically recursive}, however the first is a \emph{linear recursive} model
while the other is an \emph{iterative} model.

The two models differ in how they handle the recursive calls:
\begin{itemize}
    \item \textbf{Linear Recursion} - Each recursive call waits for the result of the next call before proceeding, 
        leading to a stack of calls (deferred operations) that can grow large for deep recursion. 
    \item \textbf{Iteration} - The function maintains a state (which is fixed) and updates it in each iteration, avoiding the need for multiple stack frames and thus being more memory efficient.
        This state provides a clear description of the state of process at any point
\end{itemize}
Crucially, languages such as JavaScript does not have this memory issue where the amount of memory grows with function calls.\\
This property is known as \emph{tail-recursive} and will execute recursive calls in constant space.

\end{document}